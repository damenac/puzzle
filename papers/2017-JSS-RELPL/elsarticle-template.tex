\documentclass[preprint,5p]{elsarticle}

%\usepackage[sc]{mathpazo}
%\usepackage{wrapfig}
%\usepackage{hyperref}
\usepackage{amsmath,amssymb,amsfonts,mathrsfs}
%\usepackage[amsmath,thmmarks]{ntheorem}
\usepackage{lineno}
\usepackage[colorlinks=true,allcolors=black,breaklinks,draft=false]{hyperref}
%\modulolinenumbers[1]

\journal{Journal of Software and Systems}

%% `Elsevier LaTeX' style
\bibliographystyle{elsarticle-num}

\begin{document}

\begin{frontmatter}

\title{Reverse Engineering Language Product Lines from Existing DSL Variants}

\author{David M\'endez-Acu\~na}
\ead{david.mendez-acuna@inria.fr}

\author{Jos\'e A. Galindo}
\ead{jagalindo@inria.fr}

\author{Benoit Combemale}
\ead{benoit.combemale@inria.fr}

\author{Arnaud Blouin}
\ead{arnaud.blouin@inria.fr}

\author{Benoit Baudry}
\ead{benoit.baudry@inria.fr}

\address{INRIA/IRISA and University of Rennes 1, France}

\begin{abstract}
The use of domain-specific languages (DSLs) has become an alternative during the development of complex systems. In this context, an emerging phenomenon is the existence of DSL variants, which are different versions of a DSL adapted to specific purposes but that still share commonalities. In such a case, the challenge for language designers is to reuse, as much as possible, previously defined language constructs to narrow implementation from scratch. To overcome this challenge, recent research in software languages engineering introduced the the notion of \textit{\textbf{language product lines}}. Similarly to software product lines, language product lines are often built through a bottom-up approach: there is a set of existing DSL variants that language designers decompose into several interdependent language modules representing the features of the product line. Besides, they define variability models that capture both commonalities and particularities of the DSL variants. In this article, we propose a reverse-engineering technique to ease-off such a development scenario. Our approach receives a set of DSL variants which are used to automatically recover a language modular design and to synthesize the corresponding variability models. The validation is performed through a case study that consists of three different variants of a DSL for finite state machines. We use our approach to reverse engineering a language product line, and then we check that the detected variation points correspond to the actual differences among the variants. This validation shows that our approach is able to correctly identify commonalities and variability. Besides, it allows us identifying open issues.

\end{abstract}

\begin{keyword}
Language product lines, software languages engineering, domain-specific languages, reverse-engineering.
\end{keyword}

\end{frontmatter}

%\linenumbers

\section{Introduction}
\label{sec:introduction}

The use of domain-specific languages (DSLs) has become a successful technique to achieve separation of concerns in the development of complex systems \cite{Cook:2006}. A DSL is a software language in which expressiveness is scoped into a well-defined domain that offers a set of abstractions (a.k.a., language constructs) needed to describe certain aspect of the system \cite{Combemale:2014}. For example, in the literature we can find DSLs for prototyping graphical user interfaces \cite{Oney:2012}, specifying security policies \cite{Lodderstedt:2002}, or performing data analysis \cite{Eberius:2012}. 

Naturally, the adoption of such language-oriented vision relies on the availability of the DSLs needed for expressing all the aspects of the system under construction \cite{Clark:2013}. This fact carries the development of many DSLs which is a challenging task due the specialized knowledge it demands. A language designer must own not only quite solid modeling skills but also the technical expertise for conducting the definition of specific artifacts such as grammars, metamodels, compilers, and interpreters. As a mater of fact, the ultimate value of DSLs has been severely limited by the cost of the associated tooling (i.e., editors, parsers, etc...) \cite{jezequel:2014}.

To improve cost-benefit when using DSLs, the research community in software languages engineering has proposed mechanisms to increase reuse during the construction of DSLs. The idea is to leverage previous engineering efforts and minimize implementation from scratch \cite{Storm:2013}. These reuse mechanisms are based on the premise that ``software languages are software too'' \cite{Favre:2011} so it is possible to use software engineering techniques to facilitate their construction \cite{Kleppe:2009}. In particular, there are approaches that take ideas from Component-Based Software Engineering (CBSE) \cite{Cleenewerck:2003} and Software Product Lines Engineering (SPLE)  \cite{Zschaler:2010} during the construction of new DSLs.

% Problem statement
A classical way for adopting the aforementioned reuse mechanisms is to group language constructs into interdependent language modules that can be later extended and/or imported as part of the specifications of future DSLs. This type of solution has ultimately gained momentum and, nowadays, there are a diversity of approaches that facilitate such modular DSLs design \cite{Vacchi:2015,Mernik:2013,Rumpe:2010}. However, the definition of language modules that can be actually useful in future DSLs is not easy. In part, this is due to the fact that the reuse of a language module implies the reuse of all the constructs it offers and language designers do not always have the information that allow them to predict the correct combination of constructs that go well together. What is the correct level of granularity? Are there constructs that should be always together? Are there constructs that should be always separated? In addition, the construction of a variability model is quite challenging as well. It requires domain knowledge.

A more pragmatical approach to leverage reuse in the construction of DSLs is to focus on legacy DSLs \cite{degueule:2015}. That is, to exploit reuse in existing DSLs that are were not necessarily built for being reused but that share some commonalities (i.e., they provide similar language constructs). Using this strategy, language designers can obtain valuable reuse information from real DSLs. For example, they can identify groups of constructs that are frequently used together. Then, a catalog of language modules can be extracted from the commonalities of the DSLs by means of reverse-engineering methods. 

In this paper, we present an approach to reverse engineering a language product line from a given set of DSLs. To to so, we first identify and extract reusable language modules. Then, we infer a variability model that represents the variability existing in the set of DSLs. This variability model can be used for configuring new DSLs. 

This paper is organized as follows: Section \ref{sec:background} introduces a set of preliminary definitions/assumptions as well as some motivating examples that we use all along the paper. Section \ref{sec:apprach} describes our approach. Section \ref{sec:validation} presents the empirical study that validates the approach. Section \ref{sec:threads} discusses the threads to validity. Section \ref{sec:relatedwork} presents the related work and, finally, Section \ref{sec:conclusions} concludes the paper. 

%The aforementioned reuse mechanisms can be adopted by means of two different approaches: \textit{top-down} and \textit{bottom-up}. The top-down approach proposes the design and implementation from scratch of reusable language modules that can be employed in the construction of several DSLs. Differently, the bottom-up approach proposes the use or reverse-engineering processes to extract those language modules from existing DSLs that share some commonalities which can be properly encapsulated. Whereas the major complexity in top-down approaches is that language designers should design language modules by trying to predict their potential reuse; bottom-up approaches do not have to deal with that problem. Rather, the extraction of the reusable language modules is based on the detection of commonalities in existing DSLs. Consequently, bottom-up approaches can be considered as promising approach and, indeed, there are already in the literature some works (e.g., \cite{vacchi:2014}) advancing towards that direction. 

%The success of this strategy relies on a set of design decisions that favor extensibility and/or genericity thus increasing the probabilities that the language modules are useful in the future. In this case, the major complexity comes from the fact that language designers do not know \textit{a priori} the needs of future DSLs. Consequently, in practice many of these building blocks are not reusable \textit{as is} and rather they require some previous adaptation.

% Contribution


%It is quite important to mention that our approach is inspired on the work of Berger et al X that presents the foundations for measuring product line-ability of families of software products in the general case. Our contributions with respect to that work are that we apply these ideas to the particular case of DSLs. Besides, at the best of our knowledge, there is not a tool implementing such ideas so we introduce a proposal. Finally, we apply our proposal in an empirical study performed on GitHub repositories where the objective is to identify families of DSLs in the Web and to evaluate its potential reuse. 

% Outline
\section{Problem Statement}
\label{sec:problemstatement}

In this section, we describe the problem addressed in this article. To this end, we first describe the development scenario behind bottom-up language product lines. Then, we define the scope of our approach. %That means that the solution presented in this article is useful when the development project satisfies the constraints described in this scenario; other situations will need other type of solutions. 

\subsection{The Development Scenario: \\ \textbf{Bottom-Up Language Product Lines}}
\label{sec:thedevelopmentscenario}

Similarly to software product lines \cite{Linden:2007}, the development process of language product lines can be divided into two phases: domain engineering and application engineering (see Fig. \ref{fig:lple-dimensions}). During the domain engineering phase, language designers build the language product line. This process includes: the design and implementation of a set of interdependent language modules that implement the language features; and the construction of variability models encoding the rules in which those features can be combined to produce valid DSL variants. During the application engineering phase, diverse DSL variants are derived according to specific needs. Such a derivation process comprises: the selection of the features to include in a given DSL variant, i.e., language configuration; and the assembly of the corresponding language modules, i.e., language modules composition. 

\begin{figure*}
\centering
\includegraphics[width=1\linewidth]{images/lple-dimensions-fig.png}
\caption{Phases of the life cycle of a language product line \label{fig:lple-dimensions}}
\end{figure*}

The top-down and bottom up approaches are two different ways to face the aforementioned phases. In the top-down approach, domain engineering is performed first and application engineering is performed afterwards. During domain analysis, language engineers use domain analysis techniques to design and implement a set of language modules and variability models from some domain knowledge owned by experts and final users. Those artifacts can be later used to configure and compose particular DSL variants in the application engineering phase.

In the bottom-up approach, the application engineering is performed first and domain engineering is performed afterwards thought reverse-engineering techniques. In this context, during the application engineering phase, language designers build an initial DSL. As the language development project evolves, some DSL variants are needed to address new requirements. Language designers create new development branches where they implement these new variants with the corresponding adaptations. At some point, language designers realize that there is potential reuse among the variants. Hence, they launch a reverse engineering process --which in this case corresponds to the domain (re)engineering phase-- where the existing DSL variants are used to build up a language product line. Using this language product line, language designers can revisit the application engineering phase in order to create new DSL variants.

\vspace{2mm}
\textit{\textbf{The clone-and-own approach.}} In this article, we are interested in bottom-up language product lines. Specifically, we propose a technique to reverse engineering language product lines from sets of existing DSL variants built through the \textit{clone-and-own} approach. With this approach, DSL variants are created by cloning existing versions of a DSL and performing certain adaptations intended to satisfy a specific purpose. While several research works have shown that the clone-and-own practice is common in software development projects \cite{Mayrand:1996,Rubin:2015}, in a recent work we provided empirical evidences that show that it is also a common practice during language development \cite{MendezAcuna:2016}. We conducted an analysis of a large pool of DSLs obtained from \texttt{GitHub}, and we detected a relevant amount of specification clones among them. You can found the sources of the experiments in a dedicated web-site\footnote{\url{http://empiricalpuzzle.weebly.com/}}.

%As a response to this phenomenon, recent research works have proposed mechanisms to reverse engineering software product lines from sets of existing software products which share code clones. This type of approaches are a clear example of bottom-up software product lines. and discussed its on software maintenance \cite{Chatterji:2016}

%Another important characteristic of the development process during the construction of bottom-up language product lines is that, in many cases, language designers use the copy/paste/modify pattern to create  individual DSLs. This approach is explained by Lopez Herrej\'on et all for the general case of software product lines.

\vspace{2mm}
\textit{\textbf{A running example.}} Suppose a team of language designers working on the construction of the DSL for finite-state machines. To this end, language designers follow the UML specification \cite{UML:2011} to define language constructs such as states, regions, transitions, and triggers. Those language constructs are specified in terms of their syntax and semantics. So, at the end of the language development process, language designers release an executable DSL which behavior complies the UML specification.

Once this first DSL is released, the language designers are asked to build a new variant which must comply the Rhapsody specification \cite{Harel:2004} (i.e., another formalism to finite state machines). This new variant shares many commonalities with UML state machines, but introduces differences at both syntax and semantics levels \cite{Crane:2007}. At this point, language designers face the challenge of reusing, as much as possible, the constructs defined in the DSL for UML state machines.

One of the possibilities for language designers is to use the clone-and-own approach by copy-pasting the specification of the first DSL in a new project, and then performing the needed adaptations. This approach permits a fast prototyping of the new DSL variant. After this process, language designers obtain two different DSLs implementing different formalisms of state machines. Those DSL variants have some commonalities among them --which are materialized in terms of cloned specification elements--. And at the same time, the DSL have some particularities that make them unique. 

Suppose now that the final users need support for other state machines formalisms. Hence, the team of language designers is asked to implement two more DSL variants: the first one complying the Stateflow specification \cite{Martaj:2010}, and the second one complying classical Harel state machines \cite{Harel:1996}. If the language designers use, again, the clone-and-own approach, then they will obtain set of four FSM-based DSL variants having specification clones among them and with certain particularities associated to the differences of each formalism. %Those variants share specification elements, and own certain particularities.

The problem that language designers face at this point is two fold. First, they will have to build a new variant of the DSL for each new FSM formalism. This becomes specially challenging when final users need to combine some specifications to define hybrid formalisms. Language designers will have to produce a new version of the DSL for each desired combination, and the clone-and-own approach becomes impractical. Second, the existence of specification clones increases the maintenance costs of the involved DSLs. If language designers detect bugs in one of the specifications, they will have to check all the DSL variants. While several approaches have been proposed to exploit the notion of code clones to produce software product lines from existing product variants \cite{LopezHerrejon:2015,Martinez:2015,Martinez:2015b}, in this article we exploit the notion of language specification clones to reverse-engineering language product lines from existing DSL variants.

\subsection{Scope of the Approach: \\ \textbf{Executable Domain Specific Languages}}
\label{sec:technologicalscope}

All the ideas presented in this article are focused to executable domain specific modeling languages (xDSMLs) where the abstract syntax is specified through \textit{metamodels}, and the dynamic semantics is specified operationally as a set of \textit{domain specific actions} \cite{Combemale:2013}. Whereas metamodels are class diagrams that represent language constructs and relationships among them, domain specific actions are Java-like methods that introduce behavior in the metaclasses of a given metamodel \cite{Jezequel:2015b}. % Such an injection is performed using a weaving process as in aspect-oriented programming \cite{Jezequel:2015b}. %Concrete syntax is out of the scope of this article. 

Fig. \ref{fig:fig-dsl-example} illustrates this type of DSLs through a simple example on finite states machines. In that case, the metamodel that implements the abstract syntax contains three metaclasses: \textsl{StateMachine}, \textsl{State}, and \textsl{Transition}. There are some references among those metaclasses representing the relationships existing among the corresponding language constructs. The domain specific actions at the right of the Fig. \ref{fig:fig-dsl-example} introduce the operational semantics to the DSL. In this example, there is one domain specific action for each metaclass. Note that the interactions among domain specific actions can be internally specified in their implementation by means of the \textit{interpreter pattern}, or externalized in a model of computation \cite{Combemale:2013}.

\begin{figure}
\centering
\includegraphics[width=1\linewidth]{images/fig-dsl-example.png}
\caption{A simple DSL for finite state machines}
\label{fig:fig-dsl-example}
\end{figure}
\section{Proposed approach}
\label{sec:approach}

\subsection{Overview}

\subsection{Breaking down the DSLs}

The objective of this step is to divide the DSL in several language modules that can be latter composed. To do so, we need to deal with two problems. The former is to find the way in which the we will split the language constructs. The second one is to define the language modules themselves with their corresponding required and provided interfaces. 

\subsubsection{Identifying constructs distribution}

The input is the set of DSLs. We first identify the language constructs for each one. Then we perform a match according to a given comparison operator. Then, we merge. Finally, we execute a graph partitioning algorithm. 

\begin{figure*}
\centering
\includegraphics[width=1\linewidth]{images/breaking-down.pdf}
\caption{Process for breaking down a set of DSLs}
\label{fig:breaking-down}
\end{figure*}

\subsubsection{Specifying language modules}

Note that both DSLs and modules are, at the end, set of language constructs. Hence, one may think that a module is also a DSL. Although this is technically true, there is a substantial difference between DSLs and modules. A DSL is a closed set of constructs that materializes a complete DSL specification that is ready to be used. Contrariwise, a module is set of constructs whose specification may depend on other constructs defined in other modules. A module can be used (and considered itself as a DSL) as long as their dependencies are fulfilled.

Accordingly, the main requirement for supporting separation of features in DSLs relies on the capability of expressing dependencies between language modules. In this context, we have identified two types of dependencies: \textit{aggregation} and \textit{extension}. In the following, we explain each of them and we present the corresponding tool support.

\textbf{Language modules \textit{aggregation}:} In aggregation, there is a \textit{requiring module} that \underline{uses} some constructs provided by a \textit{providing module}. The requiring module has a dependency relationship towards the providing one that, in the small, is materialized by the fact that some of the classes of the requiring module have references (simple references or containment references) to some constructs of the providing one.

In order to avoid direct references between modules, we introduce the notion of interfaces for dealing with modules' dependencies. In the case of aggregation, the requiring language has a \textit{required interface} whereas the providing one has the \textit{provided interface}. A required interface contains the set of constructs required by the requiring module which are supposed to be replaced by actual construct provided by other module(s).

It is important to highlight that we use \textit{model types} \cite{Steel:2007} to express both required and provided interfaces. As illustrated on top of Figure \ref{fig:approaches-interfaces}, the relationship between a module and its required interface is \textit{referencing}. A module can have some references to the constructs declared in its required interface. In turn, the relationship between a module and its provided interface is \textit{implements} (deeply explained in \cite{Degueule:2015}). A module implements the functionality exposed in its model type. If the required interface is a subtype of the provided interface, then the provided interface fulfills the requirements declared in a required interface. Note that the partial sub-typing relationship defined in \cite{Guy:2012}, permits a required interface being partially fulfilled by a provided interface. The result of the composition will be a module with a new required interface that contains only those elements that were not provided.


\textbf{Language modules \textit{extension}:} In this case, there is an extension module that (naturally) \underline{extends} the functionality provided by \textit{base module}. The extension module has a dependency to the base module. Moreover, the extension module has little sense by itself without the existence of a base module \cite{Erdweg:2012}. Note that this is a conceptual difference with respect to modules aggregation where the required make sense by itself but requires some external services in order to work correctly.

There are to different mechanisms for extending a base module: \textit{constructs specialization} and \textit{open classes} \cite{Clifton:2000}. In constructs specialization, the constructs of the base module can be extended by adding new subclasses. The extension module contains the new subclasses that reference (by means of the inheritance relationship) the constructs of the base module that are being extended. In this case, the base module remains intact in after the composition but there are additional constructs. In open classes, constructs of the base module can be re-opened and modified by the extension module. For example, for adding a new attribute to a given construct without creating a sub-class, or for overriding a given segment of the semantics. In this case, the extension module is altered after the composition phase. 

Similarly to aggregation, dependencies between the base and the extension modules are specified through interfaces. The base module exposes an \textit{extension point interface} with the constructs that can be extended. In turn, the \textit{extension interface} declares the constructs of the base module that are being extended.

Like in aggregation, and as illustrated at the bottom of Figure \ref{fig:approaches-interfaces}, we use model types of expressing these interfaces. Although the approach is quite similar, there is one fundamental difference between the interfaces in aggregation and the interfaces in extension: the relationship between an extension module and its extension interface is \textit{usage} more than just referencing. That means that the module can to reference the elements declared in the required interface and also modify them by adding new elements. This capability is introduced to support extension by the open-classes mechanism.

\begin{figure*}
\centering
\includegraphics[width=1\linewidth]{images/approaches-interfaces.pdf}
\caption{Interfaces for modularization of DSLs}
\label{fig:approaches-interfaces}
\end{figure*}

\subsection{Inferring the variability model}

After having a set of language modules with their corresponding references among them, we need to automatically infer a variability model that represents the existing variability. To do so, we use as input an algorithm that, based on the dependencies graph of the modules, infer a simple variability model. 



\subsection{Deriving a DSL}

Once the variability of the language product line is correctly specified, the next step is to configure DSLs by using the variability model. Since the variability model is expressed in CVL, the configuration of the language product corresponds to the specification of a realization model that captures the decisions made by the language designers that are configuring the DSL. Our approach uses the realization model to produce the corresponding Melange script. 

As an example, consider the configuration presented in the variability model of the Figure \ref{fig:multi-dimensional-variability}. The corresponding Melange script is presented in the following listing code snippet:

\begin{lstlisting}
language ModuleA {
   ecore MetamodelA.ecore
   with package.A.Semantics6
}

language ModuleB {
   ecore MetamodelB.ecore
   with package.B.Semantics4
}

language MyDSL {
   aggregation(ModuleB, ModuleA)
}
\end{lstlisting}

Note that the Melange script only contains the language elements that correspond to a given configuration. For example, the ModuleA contains only the Semantics6 because it was the choice made at configuration time. Similarly, ModuleB only contains Semantics4. Note also that there is a third language appearing in the script: MyDSL. This language represents the composition of the language modules and can be understood as the root of the script. In this case, this statement of Melange indicates that the modules A and B are composed by aggregation. The first element in the operation corresponds to the requiring module and the second element corresponds to the providing module. 

Once the configuration process produces a Melange script that captures the choices made by the language designers for a particular DSL, it is necessary to compose the declared language modules and produce the DSL. The composition of a set of modules requires a previous phase of compatibility checking. Not all language modules are compatible and in that case composition cannot be performed. In our approach, check the compatibility of two language modules is to verify the sub-typing relationship between the required and provided interface (for the case of aggregation), and the extension and extension point interfaces (for the case of extension).

Once this compatibility checking is correctly verified, language modules are composed. In particular, their specifications are be merged to generate a complete language specification. This merging basically replaces the elements of the required interface by its corresponding implementation in the provided component. A similar process is performed in the case of extension. 
\section{Case study: Finite State Machines}
\label{sec:validation}

 To evaluate of our approach, we use as case study the set of DSLs for state machines. Note that a simplified version have been used as running example along this paper (see Section \ref{sec:thedevelopmentscenario}). This case study is inspired from the analysis of variability on languages for finite state machines provided by Crane et al. \cite{Crane:2007}, and it is composed of three different DSLs: UML state diagrams, Rhapsody, and Harel's state charts. As aforementioned, these DSLs have some commonalities since they are intended to express the same formalism. According to the development scenario we address in this paper, these commonalities will be materialized as clones in the DSL specifications. In this section, we summarize both commonalities and differences existing in the case study. Then, we apply our approach and we present the obtained results. To explain this case stude we followed a mehod inspired by Runeson et al\cite{runeson-book}.

\subsection{Background to the research project}

\subsection{Case Study Design and planning}
\subsubsection{Rationale}
\subsubsection{Objective}
\subsubsection{Units of Analyses}
\subsubsection{Case Study protocol}

\subsection{Data collection}


\subsubsection{Description of the Commonalities}

Generally speaking, state machines are graphs where nodes represent states and arcs represent transitions between the states \cite{Harel:1987}. The execution of a state machine is performed in a sequence of \textit{steps} each of which receives a set of events that the state machine should react to. The reaction of a machine to set of events can be understood as a passage from an initial configuration (t$_i$) to a final configuration (t$_{f}$). A configuration is the set of active states in the machine.

The relationship between the state machine and the arriving events is materialized at the level of the transitions. Each transition is associated to one or more events (also called triggers). When an event arrives, the state machine fires the transitions outgoing from the states in the current configuration whose trigger matches with the event. As a result, the source state of each fired transition is deactivated whereas the corresponding target state is activated. Optionally, guards might be defined on the transitions. A transition is fired if and only if the evaluation of the guard returns true at the moment of the trigger arrival.

The initial configuration of the state machine is given by a set of initial pseudostates.  Transitions outgoing from initial pseudosates are fired automatically when the state machine is initialized. In turn, the execution of a state machine continues until the current configuration is composed only by final states (an special type of states without outgoing transitions).

All of the DSLs included in this case study support the notion of region. A state machine might be divided in several regions that are executed concurrently. Each region might have its own initial and final (pseudo)states. In addition, the DSLs also support the definition of different types of actions. States can define entry/do/exit actions, and transitions can have effect actions.

\subsubsection{Description of the Variability}

\vspace{2mm}
\textit{\textbf{Abstract syntax variability.}} Differences at the level of the abstract syntax between the DSLs under study correspond to the diversity of constructs each of those DSLs provide. In particular, there are differences in the support for transition's triggers and pseudostates.

In the case of transitions' triggers, whereas Rhapsody only supports atomic triggers, both Harel's statecharts and UML provide support for composite triggers. In Harel's statecharts triggers can be composed by using \texttt{AND}, \texttt{OR}, and \texttt{NOT} operators. In turn, in UML triggers can be composed by using the \texttt{AND} operator.

In the case of pseudostates, whereas all the DSLs support \texttt{Fork}, \texttt{Join}, \texttt{ShallowHistory}, and \texttt{Junction}, there are two psueudostates i.e., \texttt{DeepHistory} and \texttt{Choice} that are only supported by UML. The \texttt{Conditional} pseudostate is only provided by Harel's state charts. Table \ref{fig:oracle} shows the language constructs provided by each DSL.

\begin{table*}[t]
\centering
\includegraphics[width=1\linewidth]{images/tab-oracle-statemachines}
\caption{Diversity of constructs provided by the DSLs for state machines}
\label{fig:oracle}
\end{table*}

\vspace{2mm}
\textit{\textbf{Semantic variability.}} Semantic differences between the DSLs under study can be summarized in three issues:

\vspace{2mm}
\textit{(1) Events dispatching policy:} The first semantic difference in the operational semantics of state machines refers to the way in which events are consumed by the state machine. In a first interpretation, simultaneous events are supported i.e., the state machine can process more than one event in a single step. In a second interpretation, the state machine follows the principle of run to completion i.e., the state machine is able only of supporting one event by step so several events require several steps.

The semantics of UML and Rhapsody fit the run to completion policy for events dispatching whereas Harel's statecharts support simultaneous events.

\vspace{2mm}
\textit{(2) Execution order of transitions' effects:} It is possible to define actions on the transitions that will affect the execution environment where transitions are fired. These actions are usually known as transitions' effects. All the DSLs for state machines in our family support the expression of such effects. However, there are certain differences regarding their execution.

The first way of executing the effects of a transition is by respecting the order in which they are defined. This is due to the fact that transitions effects are usually defined by means of imperative action script languages where the order of the instructions is intrinsic. The second interpretation to the execution of transitions' effect is to execute them in parallel. In other words, the effects are defined as a set of instructions that will be executed at the same time so no assumptions should be made with respect to the execution order.

UML and Rhapsody execute the transition effects in parallel. Harel's statecharts execute transition effects simultaneously.

\vspace{2mm}
\textit{(3) Priorities in the transitions:} Because several transitions can be associated to the same event, there are cases in which more than one transitions are intended to be fired in the same step. In general, all the DSLs for state machines agree in the fact that all the activated transitions should be fired. However, this is not always possible because conflicts might appear. Consider the state machine presented in Fig \ref{fig:conflicting-priorities}. The transitions $T_D$ and $T_E$ are conflictive because they are activated by the same event i.e., $e_2$, they exit the same state, and they go to different target states. Then, the final configuration of the state machine will be different according to the selected transition.

\begin{figure}[h!]
  \centering
  \includegraphics[width=1\linewidth]{images/conflicting-priorities.pdf}
  \caption{Example of a state machine with conflicting priorities}
  \label{fig:conflicting-priorities}
\end{figure}

To tackle this situation, it is necessary to establish policies that permit to solve such conflicts. Specifically, we need to define a mechanism for prioritizing conflicting transitions so the interpreter is able to easily select a transition from a group of conflicting transitions. One of the best known semantic differences among DSLs for state machines is related with these policies. In particular, there are two different mechanisms for solving this kind of conflicts. A first mechanism for solving conflicting transition is to select the transition with the lower scope. That is, the deeper transition w.r.t. the hierarchy of the state machine.

In the example presented in Fig \ref{fig:conflicting-priorities} the dispatched transition according to this policy would be the transition $T_E$ so the state machine would move to the state $S_5$. The second mechanism for solving conflicts in the transition is to select the transition with the higher scope. That is, the higher transition w.r.t. the hierarchy of the state machine. In the example presented in Fig \ref{fig:conflicting-priorities} the dispatched transition according to this policy is the transition $T_D$ so the state machine will move to the state $S_4$.

The semantics of UML and Rhapsody fits on the first interpretation i.e., deepest transition priority whereas the semantics of Harel's statecharts fits on the second interpretation i.e., highest transitions priority.

%are reified by the fact that not all the DSLs have the same behavior at execution time. For example, whereas Harel's statecharts attend simultaneous events in parallel, both UML and Rhapsody follow the run to completion principle. So, simultaneous events are attended sequentially \cite{Crane:2007}. Consequently, not all the domain-specific actions are the same. In particular, the domain-specific actions \texttt{eval()} and \texttt{step()} in the \texttt{StateMachine} metaclass are different in each DSL.

\subsection{Applying our Approach}

The starting point of the applicability of our approach in the case study is a set of DSLs implementing each of the specifications explained above. Hence, at the beginning we have three different DSLs for state machines that can be accessed in a \texttt{GitHub} repository\footnote{GitHub repository for the case study: \url{https://github.com/damende/puzzle/tree/master/examples/state-machines}}. Using these specifications as input, we proceed to apply our approach.

The results are summarized in Fig. \ref{fig:results-casestudy}. At the left of the figure we present the set of language modules we obtained as well as the language interfaces existing among them. Those modules group the language constructs according to the heuristic introduced in Section \ref{sec:reverseengineeringmodules} on breaking down intersections. At the right of the figure we show the corresponding variability models. Each feature of the feature models is associated to a given language module. In turn, the semantic variability points in the orthogonal model are associated to clusters of domain specific actions.

%Note that we marked different configurations in the figure to identify each of the corresponding DSLs. In addition, we calculated the number of possible configurations. We obtained that with this variability model, we can obtain XXX DSLs for state machines.

\begin{figure*}
\centering
\includegraphics[width=1\linewidth]{images/results-casestudy.png}
\caption{Language product line produced for the case study of the finite state machines. }
\label{fig:results-casestudy}
\end{figure*}

\vspace{2mm}
\textbf{\textit{Analysis of the results.}} Let us now discuss the results of the case study. As expected, we obtained a language product product line from a set of DSL variants for finite state machines. But... Does this product line identify all the variation points and commonalities existing in the DSL variants? Are those variation points properly specified in the language modular design and variability models? Since we know these variation points and commonalities, we can check whether they are appear in the produced language product line. The results of this verification are presented in Table \ref{fig:validation-results}.

The results are promising in the case of abstract syntax variability. According to the Table \ref{fig:oracle}, the DSL variants share 17 constructs in common. Those constructs are properly factorized in a language module that we named StateMachine. This module is correctly identified during the recovering of the language modular design, and it is properly specified as a language module in terms of a metamodel enhance with domain specific actions and offering a provided interface. Besides, the particularities of the DSL variants are also well factorized. There is a module that contains the constructs NotTrigger and OrTrigger that belong only to the variant complying the Harel' statecharts specification. Besides, there are three additional modules that contain the constructs AndTrigger, Choice, and Conditional respectively. Using this modular design, we can re-compose any of the three initial DSL variants.

The situation is different for the case of semantic variability. Although our reverse-engineering strategy is able to identify that the domain specific actions are different in the three DSL variants, the level of granularity at which those variation points are detected is coarser than one might expect. At the beginning of this section, we described three semantic variation points and their possible interpretations i.e., events dispatching policy, execution order of transitions' effects, and priorities of conflicting transitions. Using the proposed technique, we can identify just one semantic variation point indicating that the language module called StateMachines contains three different clusters of domain specific actions, which is reflected in the orthogonal variability model.

This threat to validity of our technique can be explained by the fact that the analysis of commonalities and variability is conducted by means of static analysis. We can analyze the structure of the metamodels and the domain specific actions, but not their behavior at runtime. Hence, we cannot see how these differences impact the execution of the models. For example, we cannot infer that the differences among the domain specific actions in the StateMachine module impact the way in which conflicting priorities are managed. A next step in this research could be to use also dynamic analysis in the domain specific actions to better specify semantic variation points.

\begin{table}
\centering
\includegraphics[width=1\linewidth]{images/validation-results}
\caption{Analysis of the results of the case study}
\label{fig:validation-results}
\end{table}

%\section{Threads to validity}
\label{sec:limitations}
\section{Related Work}
\label{sec:relatedwork}

The idea of reverse engineering software product lines from product variants has been already studied in the literature. Besides, there are several approaches that address this issue for the case in which the product variants have been built using the clone-and-own approach \cite{LopezHerrejon:2015,Martinez:2015,Martinez:2015b}. Although the applicability of such idea to the specific case of language product lines is quite recent, there are some related work that we discuss in this section. 

\vspace{2mm}
\textbf{\textit{Recovering a language modular design.}} The first challenge during reverse engineering language of product lines is to recover a language modular design. Although this challenge has not received proper attention, we found an approach that proposes insightful advances in this direction \cite{Kuhn:2015}. In that work, the language modular design is achieved by defining one language module for each construct. That means that the reverse engineering process will result in a language product line containing as many features as constructs exist in the DSLs.

This approach permits to exploit the variability in the language product line since it provides a high level of granularity in the decomposition of language modules. Hence, language designers can make decisions with an important level of detail. However, the complexity of the product line might increase unnecessarily. From the point of view of language users, there are clusters of language constructs that always go together thus separation is not needed. For example, in our running on state machines, the concepts of \texttt{StateMachine}, \texttt{State}, and \texttt{Transition}, go always together since they correspond to a commonality of all the input DSLs. Separating these constructs in different features is not necessary in this case and this increases the complexity of the variability models. This can be a real issue if language designers decide to apply automatic analysis operations on those models.

Differently, in our approach we use the notion of specification clones and intersections in order to achieve a level of granularity that captures the variability existing in the DSL variants given in the input. This permits to identify those clusters of language constructs that go always together in the given variants. This decision simplifies the language product line in the sense that the amount of language modules is lower than in the approach by Kuhn et al., \cite{Kuhn:2015}. In doing so, we certainly reduce the possible variants that can be configured by the language product line. This issue can be considered as a threat to validity of our approach. 

\vspace{2mm}
\textbf{\textit{Synthesizing variability models.}} The synthesis of variability models has been largely studied in the literature. Some of those approaches have been adapted for the particular case of variability in the context of language product lines engineering. The approach presented in \cite{Vacchi:2014} proposes a search-based technique to find a features model that represents the variability existing in a set of language modules while optimizing an objective function. This approach uses an ontology that describes the domain concepts of the language product line. The second approach (presented in \cite{Kuhn:2015}) refines the former by removing the ontology. This improvement is motivated by the difficulty behind the construction of such ontology. Then, the authors propose to annotate the BNF-like grammar with certain information that is used to create a variability model. 

The aforementioned approaches support not only abstract syntax variability, but also concrete syntax and semantic variability. In the first case, the ontology can be used to identify all the existing syntactic and semantic variation points since it represents the domain from both the syntax and semantic point of view. In the second case, the annotations provide the expressiveness enough to address all these dimensions of the variability.  

There is, however, an important limitation in those approaches. Although at the modeling level, feature models have shown their capabilities to represent multi-dimensional variability and it has been validated for language product lines, there is not support for effectively reverse-engineering such multi-dimensional variability in the language product lines. Indeed, the solution provided by current approaches is to synthesize variability models where each feature capture both the abstract syntax of the language constructs and their semantics. Using this strategy, a language construct that has different semantics interpretations is represented as two language features. Those features have the same abstract syntax (a repeated definition of the specification) and their corresponding semantics. 

The problem with this strategy is that it couples abstract syntax variability with semantics variability, which limits multi-staged configuration. The scenario in which language designers configure only the abstract syntax, and final users configure their semantics is not supported since the configuration of the semantics depends also to configure a segment of the abstract syntax. 

We claim that, in order to facilitate multi-staged configuration, the abstract syntax variability should be defined separately from the semantic variability. The main contribution of our approach constitutes an answer to that claim. We use feature models to represent abstract syntax variability, and orthogonal variability models to represent semantics variability.  

\section{Discussion: Broadening the Spectrum}

The approach presented in this article is only useful when language designers follow the development scenario described in Section \ref{sec:thedevelopmentscenario} while using the technological space mentioned in Section \ref{sec:technologicalscope}. Along this paper we show how we can reduce maintenance costs and exploit variability if those conditions are fulfilled. In this section we open the broaden by discussing potential directions to support more diverse scenarios. 

\vspace{2mm}
\textit{Thinking outside the clone-and-own approach.} An important constraint of our approach is that it is scoped to DSLs that have been built through the clone-and-own approach. This fact permit to assume the existence of specification clones which is the backbone of our strategy for reverse engineering language modules. But... what if we have DSLs that are not necessarily built in those conditions? Suppose for example that we have as input a set of DSLs that share certain commonalities but that have been developed in different development teams. In that case, the probability of finding specification scenarios is quite reduced, and our approach will not be useful. How our strategies can be extended to deal with such a scenario?

The answer to that question relies on the definition of more complex comparison operators. As we deeply explain in Section \ref{sec:reverse-engineering}, the very first step of our reverse engineering strategy is to perform a static analysis of the given DSLs and apply two comparison in order to specify specification clones. If what we want is to find commonalities that are not necessarily materialized in specification clones but in "equivalent functionality", then we need to enhance the comparison operators in order to detect such as equivalences. 

Note the complexity behind the notion of "equivalent functionality". In the case of abstract syntax, two meta-classes might provide equivalent functionality by defining different language constructs e.g., using different names for the specification elements and even different relationships among them. In the case of the semantics, two different domain specific actions might provide equivalent functionality through different programs. We claim that further research is needed to establish this notion of equivalence thus supporting more diverse development scenarios. 

%\vspace{2mm}
%\textit{Thinking outside metamodels and domain specific actions.} Another issue to consider is the comparison of the DSLs specifications. In our case, we propose a comparison operator that is quite strict in the sense that it validates that all the specification elements are the same. This guarantees that the is actually copy paste and permits to do the division in a safe way. However, we can loose some reuse opportunities. For example, .... We claim that there is room for better developing the comparison operators. We can for example, consider issues such as .. subtyping, or even. 

%At the implementation level we provide an interface that facilitates this process. 

%\textit{On the purpose of the modular design} The reverse engineering process is guided by a clear purpose. In our case, the purpose is to reduce the maintenance costs of the DSLs. This is evidenced in the breaking down strategy which is mainly focused on removing the specification clones. A direct consequence of this is that we are able to support the variation points existing in the given set of DSLs. For example, we know that the is one language supporting AND and OR triggers. However, our approach is limited to the variation points existing in the input DSLs. If for example, we want to create a new DSL that contains only ANDTrigger, the produced language product line will not permit that. Indeed, we consider that there is room for other strategies more intended to exploit the variability. For example, in the approach presented in X the division is performed at the level of the language constructs. Then, each feature represents one construct. The limitation of this approach, however, is that the variability models might become too large and the configuration process might be tedious. Other solution can be the support for human intervention during the breaking down process. Another solution might be the optimization of some well-designed principles. Indeed, we have some preliminary experiments by using meta-heuristics to optimize high-cohesion and low-coupling. The partial results are promising but in that case the development scenarios are quite diverse. 

%It is wort mentioning that at implementation level of our approach is conceived in such a way that the modularization strategy can be easily replaced. Indeed, we provide the interface IBreaker that supposes the method break. It receives a set of DSLs and it returns the set of language modules. In doing so, we facilitate the experimentation with new strategies that can be more appropriated to a particular context. 


\section{Conclusions}
\label{sec:conclusions}

In this paper, we provide an approach for exploiting reuse during the construction of DSLs. We demonstrate that it is possible to partially automate the reuse process by identifying commonalities among DSLs and automatically extracting reusable language modules that can be later used in the construction of new DSLs. We evaluated our approach in a real industrial case study and we demonstrate that there is an important amount of potential reuse in DSLs in public repositories.

%As future work, we plan to propose approaches to automatically build language product lines i.e., software product lines where the products are DSLs. The intention is to follow with the idea of automating the reuse process. This time, using ideas that facilitate the management of the variability existing among DSLs. 

\section*{Acknowledgments}
This work is supported by the ANR INS Project GEMOC (ANR-12-INSE-0011); the bilateral collaboration VaryMDE between Inria and Thales; the bilateral collaboration FPML between Inria and DGA; and the European Union within the FP7 Marie Curie Initial Training Network “RELATE" under grant agreement number 264840. 
\section*{References}

\bibliography{mybibfile}

\end{document}
