\section{Related work}
\label{sec:relatedwork}

Software reuse has been largely studied during the last decades. The complexity behind the definition of reusable software is well-known and many approaches has been proposed to facilitate this task. The work of Caldiera and Basili \cite{Caldiera:1991} is a clear example of such approaches. It is intended to automatically extract reusable software modules from existing software systems. Then, a set of metrics for qualifying the quality of the produced modules. As mentioned in the introduction, this work has inspired the approach presented in that work paper. 

The main contribution of the research presented this paper is that we apply those ideas in the construction of DSLs. To do so, we use a strategy based on Venn diagrams that, in turn, has been inspired in the phenomenon of domains overlapping identified by V\"oelter et al \cite[p. 60-61]{voelter:2013}.

As a matter of fact, our approach is not the first one that tries to increase reuse in the construction of DSLs. The community of software languages engineering has been intensively working on this issue and noways we can find approaches for components-based languages development (such as \cite{Mernik:2013,Rumpe:2010}). In that context, our approach can be positioned as a reverse engineering technique for increasing reuse where reusable language modules are extracted from existing DSLs. It is worth noting that there are other approaches working on reverse engineering for DSLs. For example, the research presented in \cite{vacchi:2014} and \cite{Kuhn:2015}, is intended to synthesize language product lines (i.e., software product lines where the products are DSLs) from a DSL specification. 

 

