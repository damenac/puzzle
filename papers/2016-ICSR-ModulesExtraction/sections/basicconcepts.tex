\section{Preliminary Definitions}
\label{sec:background}



\vspace{-3mm}
\subsubsection{Implementation:} In order to implement a DSL, language designers need a tool set that offer capabilities to specify a DSL according to the selected technological space. This kind of tool sets are provided by language workbenches (such as Eclipse Modeling Framework or MetaEdit+) that provide meta-languages for where syntax and semantics can be expressed. The ideas presented in this paper are implemented in an Eclipse-based language workbench. In particular, metamodels are specified in the Ecore language whereas domain-specific actions are specified as methods in Xtend programming language\footnote{\url{http://www.eclipse.org/xtend/}}. The mapping between metaclasses and domain-specific actions is specified by using the notion of aspect introduced by the Kermeta 3\footnote{\url{https://github.com/diverse-project/k3/wiki/Defining-aspects-in-Kermeta-3}} and Melange\footnote{\url{http://melange-lang.org/}} as explained in \cite{degueule:2015}. 

\subsubsection{A simple DSL:} Let us illustrate this idea by using a simple example. Consider the metamodel introduced at the top of Figure \ref{fig:k3-example}.

As said above, we use Melange to integrate and execute the definitions of the abstract syntax and semantics of a DSL. Melange is a language for modeling in the large that facilitates the integration of the different artifacts that compose the specification of a DSL. Listing \ref{lst:fsm} illustrates the use of Melange. At the left we have an abstract representation of a language that is composed of a metamodel and three aspects implementing the semantics. At the right of the figure we have the corresponding Melange script.
 
\vspace{4mm}
\begin{lstlisting}[caption=Melange script for a simple FSM language, label=lst:fsm]
language FSM {
    syntax "fsm.mm/models/fsm.ecore"
    
    with fsm.sem.StateMachineAspect
    with fsm.sem.StateAspect
    with fsm.sem.TransitionAspect
}
\end{lstlisting}

%\begin{figure}
%\centering
%\includegraphics[width=0.8\linewidth]{images/module-melange}
%\caption{Using Melange for weaving metamodels and aspects}
%\label{fig:module-melange}
%\end{figure}

\subsection{Overlapping in DSLs}

In \cite[p. 60-61]{voelter:2013}, V\"oelter et al  that observes that although many of the existing DSLs are completely different and tackle independent domains; there are related DSLs with overlapping domains. That is, they share certain language constructs i.e., they have \textbf{commonalities} between them. If two DSLs have commonalities and they are specified in the same technological space and using compatible language workbenches, then there is \textbf{potential reuse} since the specification of those shared constructs can be specified once and reused in the two DSLs \cite[p. 60-61]{voelter:2013}.

Naturally, commonalities can be found not only at the level of the syntax but also at the level of the semantics. For the technological space discussed in this paper, syntactic commonalities appear where DSLs share some metaclasses and semantic commonalities appear where DSLs share some domain-specific actions.

\subsection{Illustrating Scenario}

Let us now introduce a toy \todo[inline]{esto de toy es para tirar tu trabajo por tierra?}example to illustrate the concepts introduced so far. Consider the following a set of three DSLs that contains the language for finite state machines introduced before, and two additional ones; a DSL for Logo and another one for Flowchart. Logo is a DSL for expressing movements of the classical Logo turtle used in elementary schools for teaching the first foundations of programming. This DSL offers the constructs for moving a turtle forward and backward, as well for rotating the turtle at the left or at the right. In addition, the DSL offers simple arithmetic expressions for indicating the distance/angles the must should move/rotate. In turn, Flowchart is a DSL for expressing simple flow diagrams. Each flow is a sequence of nodes and arcs between them. An arc can be either an action or a decision. An action is a set of instructions that modify some variables in the execution context. A decision is a bifurcation point where depending on a given condition the flow goes from a direction or another. 

As a matter of fact, these DSLs are essentially different. Each of them is focus on a particular domain and offers different language constructs. However, there are syntactic and semantic commonalities that are illustrated in Figure \ref{fig:domains}. All the thee DSLs offer some expressions for modifying variables. In the case of FSM these actions are needed the specification of the actions in the states; in the case of Logo expressions are needed to specity the movement and rotation parameters; and in the case of flowchart expressions are needed to specify the body of actions. In addition, both FSM and Flowchart rely on a constraints language. The former for expressing guards in the transitions and the later for expressing guards of the decisions.

Note that each DSL is specified in terms of a set of metaclasses (top of the figure), and a set of aspects (bottom of the figure) that weave some domain-specific actions to the metaclasses. In the case of this example, the semantics of the metaclasses expression and constraints are also shared. That means that the semantics are the same. 

\begin{figure}
\centering
\includegraphics[width=1\linewidth]{images/domains-fig.pdf}
\caption{Commonalities between domains and potential reuse}
\label{fig:domains}
\end{figure}

%Commonalities can be found between two ore more DSLs of the input set. That is, we can find metaclasses and domain specific actions that are shared by more than two DSLs. Hence, intersections should be searched among all the possible combinations of the DSLs in the input set. Once those functions are defined and implemented, the second phase is to use them in order to find the intersections among the DSLs of the input set. 

It is worth to mention that the fact that two metaclasses are shared does not imply that all their domain specific actions are the same. We refer to that phenomenon as \textbf{semantical variability}. There are two constructs that share the syntax but that differ in their semantics. In such case, there is potential reuse at the level of the syntax since the metaclass can be defined once and reused in the DSLs but the semantics should be defined differently for each DSLs. 

%there are three DSLs DSLs that are totally independent. That means that they do not share any of their language constructs, and consequently, there is not potential reuse between them. Differently, the two DSLs shown at the right of the figure have overlapping domains. That means that there are a subset of language that are \large\textbf{``equal'' }\normalsize in both DSLs. Note that if two language constructs are the same, we can assume that their specifications are equal and can be reused instead of being replicated.




%Moreover, there are set of DSLs for which the domains can be hierarchically organized \cite[p. 60-61]{voelter:2013}.

%\subsection{Equivalence between language constructs}

%So far, we have based the notion of potential reuse in DSLs on the commonalities existing in a set of DSLs. Nevertheless, this assumption supposes that we are able to compare two language constructs in order to know if they are equivalent. So, now we need to define this \textit{equivalence} relationship. In particular, the comparison of two language constructs relies on two dimensions: (1) comparison of the meta-classes in the abstract syntax; and (2) comparison of the domain-specific actions in the semantics.

