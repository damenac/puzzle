\documentclass[runningheads,a4paper]{llncs}

\usepackage{amssymb}
\usepackage{amsmath}
\usepackage{graphicx}
\usepackage[applemac]{inputenx} % This is required to support characters in spanish
\usepackage[T1]{fontenc} % This is required to have small-bold-capitals
\usepackage{fancyhdr} % This is required to have nice headers and footers
\usepackage{framed} % This is required to have framed comments
\usepackage{color} % This defines colors that can be used in the text
\usepackage{soul} % This allows the usage of \hl to highlight text
\usepackage{tocloft} % This is required to have custom lists of .....
\usepackage{comment} % This is required to remove specific environments
\usepackage{caption}
\usepackage{ifthen}
\usepackage{array}
\usepackage{wrapfig}
\usepackage{hyperref}
\usepackage{pifont}
\usepackage{listings}
\usepackage{syntax}
\usepackage{graphicx}
\usepackage{url}
\usepackage{latexsym}
\usepackage{fancybox}
\usepackage{pifont}
\usepackage{lineno}
\usepackage{todonotes}
\usepackage{multirow}
\usepackage{multicol}
\usepackage{xspace}
\usepackage{setspace}
\usepackage{todonotes}
\usepackage{bbding}
\usepackage{xcolor,colortbl}
\setcounter{tocdepth}{3}
\newtheorem{mydef}{Definition}


\newcommand{\mc}[2]{\multicolumn{#1}{c}{#2}}
\definecolor{Gray}{gray}{0.9}
\definecolor{LightGray}{gray}{0.98}

\newcolumntype{a}{>{\columncolor{Gray}}c}
\newcolumntype{b}{>{\columncolor{white}}c}

\newcommand\etal[0]{\emph{et al.}\xspace}
\newcommand\td[1]{\todo[inline]{#1}\xspace}

\setcounter{tocdepth}{3}


\urldef{\mailsa}\path|{david.mendez-acuna,jagalindo,benoit.combemale,arnaud.blouin,benoit.baudry}@inria.fr|    
\newcommand{\keywords}[1]{\par\addvspace\baselineskip
\noindent\keywordname\enspace\ignorespaces#1}

\begin{document}

\mainmatter  % start of an individual contribution

\title{Identifying Reusable Language Modules in Domain-Specific Languages}
\titlerunning{Identifying Reusable Language Modules in Domain-Specific Languages}


\author{David M\'endez-Acu\~na \and Jos\'e A. Galindo \and Benoit Combemale \and \\ Arnaud Blouin \and Benoit Baudry}
\institute{University of Rennes 1, INRIA/IRISA. France\\
\vspace{1mm}\mailsa}
\authorrunning{David M\'endez-Acu\~na et. al}

\maketitle

\begin{abstract} 

The use of domain-specific languages (DSLs) has become a successful technique in the implementation of complex systems. However, the construction of this type of languages is time-consuming and requires highly-specialized knowledge and skills. Hence, researchers are currently seeking approaches to leverage reuse during the DSLs development in order to minimize implementation from scratch. An important step towards achieving this objective is to identify commonalities among existing DSLs. These commonalities constitute potential reuse that can be exploited by using reverse-engineering methods. In this paper, we present an approach intended to identify sets of DSLs with potential reuse. We also provide a mechanism that allows language designers to measure such potential reuse in order to objectively evaluate whether it is enough to justify the applicability of a given reverse-engineering process. We validate our approach by evaluating a large amount of DSLs we take from public \texttt{GitHub} repositories.

\end{abstract}

\section{Introduction}
\label{sec:introduction}

The use of domain-specific languages (DSLs) has become a successful technique to achieve separation of concerns in the development of complex systems \cite{Cook:2006}. A DSL is a software language in which expressiveness is scoped into a well-defined domain that offers a set of abstractions (a.k.a., language constructs) needed to describe certain aspect of the system \cite{Combemale:2014}. For example, in the literature we can find DSLs for prototyping graphical user interfaces \cite{Oney:2012}, specifying security policies \cite{Lodderstedt:2002}, or performing data analysis \cite{Eberius:2012}. 

Naturally, the adoption of such language-oriented vision relies on the availability of the DSLs needed for expressing all the aspects of the system under construction \cite{Clark:2013}. This fact carries the development of many DSLs which is a challenging task due the specialized knowledge it demands. A language designer must own not only quite solid modeling skills but also the technical expertise for conducting the definition of specific artifacts such as grammars, metamodels, compilers, and interpreters. As a mater of fact, the ultimate value of DSLs has been severely limited by the cost of the associated tooling (i.e., editors, parsers, etc...) \cite{jezequel:2014}.

To improve cost-benefit when using DSLs, the research community in software languages engineering has proposed mechanisms to increase reuse during the construction of DSLs. The idea is to leverage previous engineering efforts and minimize implementation from scratch \cite{Storm:2013}. These reuse mechanisms are based on the premise that ``software languages are software too'' \cite{Favre:2011} so it is possible to use software engineering techniques to facilitate their construction \cite{Kleppe:2009}. In particular, there are approaches that take ideas from Component-Based Software Engineering (CBSE) \cite{Cleenewerck:2003} and Software Product Lines Engineering (SPLE)  \cite{Zschaler:2010} during the construction of new DSLs.

% Problem statement
A classical way for adopting the aforementioned reuse mechanisms is to group language constructs into interdependent language modules that can be later extended and/or imported as part of the specifications of future DSLs. This type of solution has ultimately gained momentum and, nowadays, there are a diversity of approaches that facilitate such modular DSLs design \cite{Vacchi:2015,Mernik:2013,Rumpe:2010}. However, the definition of language modules that can be actually useful in future DSLs is not easy. In part, this is due to the fact that the reuse of a language module implies the reuse of all the constructs it offers and language designers do not always have the information that allow them to predict the correct combination of constructs that go well together. What is the correct level of granularity? Are there constructs that should be always together? Are there constructs that should be always separated? In addition, the construction of a variability model is quite challenging as well. It requires domain knowledge.

A more pragmatical approach to leverage reuse in the construction of DSLs is to focus on legacy DSLs \cite{degueule:2015}. That is, to exploit reuse in existing DSLs that are were not necessarily built for being reused but that share some commonalities (i.e., they provide similar language constructs). Using this strategy, language designers can obtain valuable reuse information from real DSLs. For example, they can identify groups of constructs that are frequently used together. Then, a catalog of language modules can be extracted from the commonalities of the DSLs by means of reverse-engineering methods. 

In this paper, we present an approach to reverse engineering a language product line from a given set of DSLs. To to so, we first identify and extract reusable language modules. Then, we infer a variability model that represents the variability existing in the set of DSLs. This variability model can be used for configuring new DSLs. 

This paper is organized as follows: Section \ref{sec:background} introduces a set of preliminary definitions/assumptions as well as some motivating examples that we use all along the paper. Section \ref{sec:apprach} describes our approach. Section \ref{sec:validation} presents the empirical study that validates the approach. Section \ref{sec:threads} discusses the threads to validity. Section \ref{sec:relatedwork} presents the related work and, finally, Section \ref{sec:conclusions} concludes the paper. 

%The aforementioned reuse mechanisms can be adopted by means of two different approaches: \textit{top-down} and \textit{bottom-up}. The top-down approach proposes the design and implementation from scratch of reusable language modules that can be employed in the construction of several DSLs. Differently, the bottom-up approach proposes the use or reverse-engineering processes to extract those language modules from existing DSLs that share some commonalities which can be properly encapsulated. Whereas the major complexity in top-down approaches is that language designers should design language modules by trying to predict their potential reuse; bottom-up approaches do not have to deal with that problem. Rather, the extraction of the reusable language modules is based on the detection of commonalities in existing DSLs. Consequently, bottom-up approaches can be considered as promising approach and, indeed, there are already in the literature some works (e.g., \cite{vacchi:2014}) advancing towards that direction. 

%The success of this strategy relies on a set of design decisions that favor extensibility and/or genericity thus increasing the probabilities that the language modules are useful in the future. In this case, the major complexity comes from the fact that language designers do not know \textit{a priori} the needs of future DSLs. Consequently, in practice many of these building blocks are not reusable \textit{as is} and rather they require some previous adaptation.

% Contribution


%It is quite important to mention that our approach is inspired on the work of Berger et al X that presents the foundations for measuring product line-ability of families of software products in the general case. Our contributions with respect to that work are that we apply these ideas to the particular case of DSLs. Besides, at the best of our knowledge, there is not a tool implementing such ideas so we introduce a proposal. Finally, we apply our proposal in an empirical study performed on GitHub repositories where the objective is to identify families of DSLs in the Web and to evaluate its potential reuse. 

% Outline
\section{Preliminary Definitions}
\label{sec:background}



\vspace{-3mm}
\subsubsection{Implementation:} In order to implement a DSL, language designers need a tool set that offer capabilities to specify a DSL according to the selected technological space. This kind of tool sets are provided by language workbenches (such as Eclipse Modeling Framework or MetaEdit+) that provide meta-languages for where syntax and semantics can be expressed. The ideas presented in this paper are implemented in an Eclipse-based language workbench. In particular, metamodels are specified in the Ecore language whereas domain-specific actions are specified as methods in Xtend programming language\footnote{\url{http://www.eclipse.org/xtend/}}. The mapping between metaclasses and domain-specific actions is specified by using the notion of aspect introduced by the Kermeta 3\footnote{\url{https://github.com/diverse-project/k3/wiki/Defining-aspects-in-Kermeta-3}} and Melange\footnote{\url{http://melange-lang.org/}} as explained in \cite{degueule:2015}. 

\subsubsection{A simple DSL:} Let us illustrate this idea by using a simple example. Consider the metamodel introduced at the top of Figure \ref{fig:k3-example}.

As said above, we use Melange to integrate and execute the definitions of the abstract syntax and semantics of a DSL. Melange is a language for modeling in the large that facilitates the integration of the different artifacts that compose the specification of a DSL. Listing \ref{lst:fsm} illustrates the use of Melange. At the left we have an abstract representation of a language that is composed of a metamodel and three aspects implementing the semantics. At the right of the figure we have the corresponding Melange script.
 
\vspace{4mm}
\begin{lstlisting}[caption=Melange script for a simple FSM language, label=lst:fsm]
language FSM {
    syntax "fsm.mm/models/fsm.ecore"
    
    with fsm.sem.StateMachineAspect
    with fsm.sem.StateAspect
    with fsm.sem.TransitionAspect
}
\end{lstlisting}

%\begin{figure}
%\centering
%\includegraphics[width=0.8\linewidth]{images/module-melange}
%\caption{Using Melange for weaving metamodels and aspects}
%\label{fig:module-melange}
%\end{figure}

\subsection{Overlapping in DSLs}

In \cite[p. 60-61]{voelter:2013}, V\"oelter et al  that observes that although many of the existing DSLs are completely different and tackle independent domains; there are related DSLs with overlapping domains. That is, they share certain language constructs i.e., they have \textbf{commonalities} between them. If two DSLs have commonalities and they are specified in the same technological space and using compatible language workbenches, then there is \textbf{potential reuse} since the specification of those shared constructs can be specified once and reused in the two DSLs \cite[p. 60-61]{voelter:2013}.

Naturally, commonalities can be found not only at the level of the syntax but also at the level of the semantics. For the technological space discussed in this paper, syntactic commonalities appear where DSLs share some metaclasses and semantic commonalities appear where DSLs share some domain-specific actions.

\subsection{Illustrating Scenario}

Let us now introduce a toy \todo[inline]{esto de toy es para tirar tu trabajo por tierra?}example to illustrate the concepts introduced so far. Consider the following a set of three DSLs that contains the language for finite state machines introduced before, and two additional ones; a DSL for Logo and another one for Flowchart. Logo is a DSL for expressing movements of the classical Logo turtle used in elementary schools for teaching the first foundations of programming. This DSL offers the constructs for moving a turtle forward and backward, as well for rotating the turtle at the left or at the right. In addition, the DSL offers simple arithmetic expressions for indicating the distance/angles the must should move/rotate. In turn, Flowchart is a DSL for expressing simple flow diagrams. Each flow is a sequence of nodes and arcs between them. An arc can be either an action or a decision. An action is a set of instructions that modify some variables in the execution context. A decision is a bifurcation point where depending on a given condition the flow goes from a direction or another. 

As a matter of fact, these DSLs are essentially different. Each of them is focus on a particular domain and offers different language constructs. However, there are syntactic and semantic commonalities that are illustrated in Figure \ref{fig:domains}. All the thee DSLs offer some expressions for modifying variables. In the case of FSM these actions are needed the specification of the actions in the states; in the case of Logo expressions are needed to specity the movement and rotation parameters; and in the case of flowchart expressions are needed to specify the body of actions. In addition, both FSM and Flowchart rely on a constraints language. The former for expressing guards in the transitions and the later for expressing guards of the decisions.

Note that each DSL is specified in terms of a set of metaclasses (top of the figure), and a set of aspects (bottom of the figure) that weave some domain-specific actions to the metaclasses. In the case of this example, the semantics of the metaclasses expression and constraints are also shared. That means that the semantics are the same. 

\begin{figure}
\centering
\includegraphics[width=1\linewidth]{images/domains-fig.pdf}
\caption{Commonalities between domains and potential reuse}
\label{fig:domains}
\end{figure}

%Commonalities can be found between two ore more DSLs of the input set. That is, we can find metaclasses and domain specific actions that are shared by more than two DSLs. Hence, intersections should be searched among all the possible combinations of the DSLs in the input set. Once those functions are defined and implemented, the second phase is to use them in order to find the intersections among the DSLs of the input set. 

It is worth to mention that the fact that two metaclasses are shared does not imply that all their domain specific actions are the same. We refer to that phenomenon as \textbf{semantical variability}. There are two constructs that share the syntax but that differ in their semantics. In such case, there is potential reuse at the level of the syntax since the metaclass can be defined once and reused in the DSLs but the semantics should be defined differently for each DSLs. 

%there are three DSLs DSLs that are totally independent. That means that they do not share any of their language constructs, and consequently, there is not potential reuse between them. Differently, the two DSLs shown at the right of the figure have overlapping domains. That means that there are a subset of language that are \large\textbf{``equal'' }\normalsize in both DSLs. Note that if two language constructs are the same, we can assume that their specifications are equal and can be reused instead of being replicated.




%Moreover, there are set of DSLs for which the domains can be hierarchically organized \cite[p. 60-61]{voelter:2013}.

%\subsection{Equivalence between language constructs}

%So far, we have based the notion of potential reuse in DSLs on the commonalities existing in a set of DSLs. Nevertheless, this assumption supposes that we are able to compare two language constructs in order to know if they are equivalent. So, now we need to define this \textit{equivalence} relationship. In particular, the comparison of two language constructs relies on two dimensions: (1) comparison of the meta-classes in the abstract syntax; and (2) comparison of the domain-specific actions in the semantics.


%\section{Motivating scenario}
\label{sec:example}

In this section we present an example that we use to illustrate the basic concepts introduced in the previous section and that we will use to present \textsc{PuzzleMetrics}. Fig. \ref{fig:motivating-example}.

Our motivation scenario is composed of three languages: logo program, state charts, and flowchart. As shown in the figure, each language is in an indepentendent package because it was implemented in isolation. 

\begin{figure}
\centering
\includegraphics[width=1.05\linewidth]{images/motivating-example.pdf}
\caption{Different relationships between DSL domains}
\label{fig:motivating-example}
\end{figure}
\section{Proposed approach}

Given a set of existing DSLs (that we term as the \textit{input set}) our approach is intended to identify commonalities --and so, potential reuse--. Then, we evaluate those commonalities in order to know if the input set is a good candidate to a reverse engineering method that permits to exploit the existing potential reuse. The reminder of this section explain how we tackle this problem.

\subsection{Identifying commonalities}
\label{sec:metrics}

In the first part of our approach, we perform static analysis in syntax and semantics of a given set of DSLs in order to build a pair of Venn diagrams such as the presented in the previous section (Figure \ref{fig:domains}). We consider these diagrams as a useful mechanism that allows language designers to easily visually identify commonalities. To this end, we designed an algorithm that is able to compute the all intersections among the syntax of the DSLs in the input set. We do the proper for the case of the domain-specific actions. 

Our algorithm for detecting \textbf{syntactic intersections} can be described as by the function that receives a set of metamodels (one for each DSL of the input set) and returns a set of tuples containing all the intersections among these metamodels. As mentioned before, there can be intersections among any of the combinations of the input set. Hence, in the result there is a tuple for each of the possible combinations of the input metamodels (i.e., the power set). Similarly, our algorithm for detecting \textbf{semantic intersections} can be described as a function that receives a set of aspects (one for each DSL of the input set) and returns a set of tuples containing all the intersections among these aspects. 

%\begin{equation}
%  Venn_{syn} : set(MM) \rightarrow set(<set(MM),set(MC)>)
%\end{equation}

%\begin{equation}
%  Venn_{syn}(mms) = \{<x,y> \mid x \in \mathcal{P}(mms), y = I_{syn}(x)\}
%\end{equation}

%Note that our algorithm relies on a function $I_{syn}$ that computes the intersection existing withing a given set of metamodels. It can be formalized as follows:

%\begin{equation}
%  I_{syn} : set(MM) \rightarrow set(MC)
%\end{equation}
%\vspace{-2mm}
%\begin{equation}
%  I_{syn}(mms) = \bigcap _{i=0}^{|mms|}mms_i
%\end{equation}

%Similarly, our algorithm for detecting \textbf{semantic intersections} can be described as a function that receives a set of aspects (one for each DSL of the input set) and returns a set of tuples containing all the intersections among these aspects. 

%\begin{equation}
%  Venn_{sem} : set(A) \rightarrow set(<set(A),set(DSA)>)
%\end{equation}

%\begin{equation}
%  Venn_{syn}(mms) = \{<x,y> \mid x \in \mathcal{P}(mms), y = I_{sem}(x)\}
%\end{equation}
%\vspace{2mm}

%This time, the algorithm for semantic commonalities relies on a function $I_{sem}$ that computes the intersection existing withing a given set of aspects. It can be formalized as follows:

%\begin{equation}
%  I_{sem} : set(A) \rightarrow set(DSA)
%\end{equation}
%\vspace{-2mm}
%\begin{equation}
%  I_{sem}(dsas) = \bigcap _{i=0}^{|dsas|}dsas_i
%\end{equation}

\subsubsection{Comparison operators:} A syntactic intersection is a set of metaclasses that are equal in two or more DSLs. Similarly, a semantic intersection is a set of domain-specific actions that are equal in two or more DSLs. At this point we need to clearly define the notion of equality between metaclasses and domain-specific actions. That is, we need to establish the criteria under we consider that two metaclasses/domain-specific actions are equal.

\begin{itemize}
\item \textbf{Comparison of metaclasses:} The name of a metaclass usually corresponds to a word that evokes the domain concept the metaclass represents. Thus, intuitively one can think that a first approach to compare meta-classes is by comparing their names. As we will see later in this paper, this approach results quite useful and it is quite probable that, we can find potential reuse.

%\begin{equation}
%  \doteq~: MC \times MC \rightarrow bool
%\end{equation}
%\vspace{-1mm}
\begin{equation}
\begin{split}
  MC_{A} \doteq MC_{B} &= true \implies \\
   & MC_{A}.name = MC_{B}.name
 \end{split}
\end{equation}

\vspace{1mm}
\hspace{3mm} Unfortunately, comparison of metaclasses by using only their names might have some problems. There are cases in which two meta-classes with the same name are not exactly the same since they do not represent the same domain concept or because there are domains that use similar vocabulary. In such cases, an approach that certainly helps is to compare meta-classes not only by their names but also by their attributes and references. Hence, we define a second comparison operator for metaclasses i.e., $\doteqdot $.

%\begin{equation}
%  \doteqdot~: MC \times MC \rightarrow bool
%\end{equation}
%\vspace{-1mm}
\begin{equation}
\begin{split}
  MC_{A} \doteqdot MC_{B} &= true \implies \\
   & MC_{A} \doteq = MC_{B} ~ \wedge \\
   & \forall a_1 \in MC_{A}.attr \mid (\exists a_2 \in MC_{B}.attr \mid a_1 = a_2) ~ \wedge \\
   & \forall r_1 \in MC_{A}.refs \mid (\exists r_2 \in MC_{B}.refs \mid r_1 = r_2)
  \end{split}
\end{equation}

\vspace{2mm}
\hspace{3mm} Although this second approach might be too restrictive, it implies that the specification of the two meta-classes are exactly the same so potential reuse is guaranteed. At the implementation we provide support for the two comparison approaches explained above. However, additional comparison operators such as the surveyed in \cite{Lafi:2011} can be easily incorporated.

\vspace{1mm}

\item \textbf{Comparing domain-specific actions:} Like methods in Java, domain-specific actions have a signature that specifies its contract (i.e., return type, visibility, parameters, name, and so on), and a body where the behavior is actually implemented. In that sense, the comparison of two domain-specific actions can be performed by checking if their signatures are equal. This approach is practical and also reflects potential reuse; one might think that the probability that two domain-specific actions with the same signatures are the same is elevated.

%\begin{equation}
%  \circeq~: DSA \times DSA \rightarrow bool
%\end{equation}
%\vspace{-1mm}
\begin{equation}
\begin{split}
  DSA_{A} & \circeq DSA_{B} = true \implies \\
   & DSA_{A}.name = DSA_{B}.name ~ \wedge \\
   & DSA_{A}.returnType = DSA_{B}.returnType ~ \wedge \\
   & DSA_{A}.visibility = DSA_{B}.visibility ~ \wedge \\
   & \forall p_1 \in DSA_{A}.params \mid (\exists p_2 \in DSA_{B}.params \mid p_1 = p_2)
 \end{split}
\end{equation}

\vspace{2mm}
\hspace{3mm} However, as the reader might imagine, there are cases in which signatures comparison is not enough. Two domain-specific actions defined in different DSLs can perform different computations even if they have the same signatures. As a result, a second approach relies in the comparison of the bodies of the domain-specific actions. Note that such comparison can be arbitrary difficult. Indeed, if we try to compare  the behavior of the actions we will have to deal with the semantic equivalence problem that, indeed, is known as be undecidable \cite{Lucanu:2013}. In this case, we a conservative approach is to compare only the structure (abstract syntax tree) body of the domain-specific action. To this end, we use the API for java code comparison proposed in \cite{Biegel:2010}. 

%\begin{equation}
%  \triangleq~: DSA \times DSA \rightarrow bool
%\end{equation}
%\vspace{-1mm}
\begin{equation}
\begin{split}
  DSA_{A} \triangleq DSA_{B} & = true \implies \\
   & DSA_{A} \circeq DSA_{B} ~ \wedge \\
   & DSA_{A}.AST = DSA_{B}.AST
 \end{split}
\end{equation}
\end{itemize}


%It is worth nothing that there is this phenomenon of \textit{semantical variability}. A necessary condition to decide whether two language constructs are equivalent is that both, the metaclass and the associated domain-specific actions are equivalent. This condition guarantees that the specification is the same not only at the level of the abstract syntax but also at the level of the semantics. However, there is a phenomenon in the literature that corresponds to semantical variability \cite{Cengarle:2009}. There is semantical variability when there there are two constructs that have the same abstract syntax (i.e., their metaclasses are equal) but that differ in the domain-specific actions. This case is of interest for us because even in the presence of semantical variability we can have some potential reuse. If the metaclasses of two constructs are the same we can reuse them even if their domain-specific actions are different. 

\vspace{-2mm}
\subsubsection{Visualizing the results:} Figure \ref{fig:shape} shows the Venn Diagram for the case of our motivating scenario. In that figure we can see that the family is an overlapping family in terms of the abstract syntax. In the case of the semantics the results are quite interesting. Note that depending on the type of comparison operator we have different results. When the comparison operator is the naming, we have the same overlapping shape that in the case of the abstract syntax. However, when the operators become more restrictive the overlapping region is reduced. 

\begin{figure}
\centering
\includegraphics[width=1\linewidth]{images/domains-inaction.pdf}
\caption{Visualizing family's shape according to the selected comparison operator}
\label{fig:shape}
\end{figure}

%\vspace{-2mm}
%\subsubsection{Visualizing semantical variability:} Note that the phenomenon of semantical variability is evident in the example presented. Where there are syntactic commonalities between DSLs Logo and FSM, there are not semantic commonalities. As an additional feature of our approach, we provide a visualization of the semantical variability phenomenon. The idea is that language designers can see what are the variations in the domain specific actions.

\subsection{Objectively evaluating the potential reuse}

The second part of our analysis corresponds to a quantitative evaluation of potential reuse. To do so, we include the reuse metrics presented in \cite{Berger:2014,Berger:126283} and we adapt them for the case of domain-specific languages. Those metrics (graphically shown in Figure \ref{fig:metrics}) are intended to measure the size of the commonalities existing among the DSLs of the input set. In this section we present these metrics in terms of the formulas we used to compute them. It is important to remember that the results provided by those metrics also depend on the comparison operator.

\begin{figure}
\centering
\includegraphics[width=1\linewidth]{images/metrics.pdf}
\caption{Metrics for evaluation of potential reuse}
\label{fig:metrics}
\end{figure}

\begin{itemize}
\item \textbf{Size of Commonality (SoC):} This metric shows the size of the core with respect to the rest of the family. It is calculated as the percentage of constructs/methods that are included in the core with respect to the union of the constructs/methods of all the DSLs of the family. 

\hspace{3mm} On the other hand, the larger the core the smaller the variability. So the amount of required decomposition is reduced and the variability model is simpler. So, one may think that a family where the core is big is a family where the variability is easier to manage. 

\vspace{1mm}
\item \textbf{Product-Related Reusability ($PRR_i$):}
This metric shows the percentage of reuse of each DSL with respect the core. Concretely, it shows for each product the amount of constructs/methods that are included in the core.

\hspace{3mm} This metric is important because we can detect the product more related to the core. This identification will be helpful at the moment of defining that core. 

\vspace{1mm}
\item \textbf{Individualization Ratio ($IR_i$):}
This metric shows the percentage of reuse of each DSL with respect the rest of the family. Concretely, it shows for each product the amount of constructs/methods that are included in at least another DSL that is member of the family.

\hspace{3mm} This metric is important because it allows the identification of the most isolated product as well as the most integrated to the family.

\vspace{2mm}
\item \textbf{Pairwise Relationship Ratio ($PWRR_{(i,j)}$):} 
This metric shows the percentage of reuse of each DSL with respect the each of the other DSLs that are members of the family. Concretely, it shows, for each product, the amount of constructs/methods that are included each of the other members of the family.
\end{itemize}

\subsubsection{Tool support.} Figure x shows the output of our tool. In this case, it receives a set of DSLs and it computes all the metrics explained above. 

\section{Evaluation}
\label{sec:validation}

In this section we present the validation of our approach. As aforementioned, this validation is twofold. On one hand, we demonstrate that our approach is correct. To do so, we take use a case study that is well documented in \cite{Crane:2007} and where we exactly know the commonalities existing among the input set. We execute our approach and we compare the result. The second part of the evaluation corresponds to demonstrate that our approach is relevant. That is, we demonstrate with empirical data that the phenomenon of syntactic and semantic commonalities is currently appearing in real DSLs and, so, our approach is relevant. 

\subsection{Evaluating correctness: The state machines case study}

The case study is composed of three different DSLs for expressing state machines: UML state diagrams, Rhapsody, and Harel's state charts. Because the three DSLs are intended to express the same formalism, they have commonalities. For example, all of them provide the basic concepts of State and Transition. However, not all those DSLs are exactly the same. In fact, they have some syntactic and semantic differences. As an example of syntactic particularities consider the case of the pseudostates. As an example of semantic differences consider the case of ... Whereas... 

\subsubsection{The oracle.} We are interested on this case study because it corresponds to a set of DSLs where the commonalities and differences are quite well documented. Hence, if we implement those DSLs by strictly following such documentation, we have an oracle to test the correctness of our approach. 

\begin{figure}
\centering
\includegraphics[width=1\linewidth]{images/oracle.pdf}
\caption{Oracle for evaluation of correctness}
\label{fig:oracle}
\end{figure}

Figure \ref{fig:oracle} shows a table with the constructs contained in the case study. Not all the DSLs have exactly the same constructs. For example, while constructs such as StateMachine, Region, State, and Transitions are shared by all the DSLs, the peudostates, the triggers, are different for each case. 

\subsubsection{Results.} The figure \ref{fig:puzzle-overlapping} shows the results of the first part of the analysis performed by our approach. It presents the Venn diagram produced for the case study of the state machines. As you can see at the left, the numbers correspond to the oracle presented in table X. The case of the semantics is revealing as well. 

\begin{figure}
\centering
\includegraphics[width=1\linewidth]{images/puzzle-overlapping.pdf}
\caption{Results for the state machines case study: identifying overlapping}
\label{fig:puzzle-overlapping}
\end{figure}

In turn, Figure \ref{fig:puzzle-modularization} shows the results for the second  and third part of the approach: identifying and extracting reusable language modules. As the reader may expect, there is a core module that contains all those language constructs that are shared by the three DSLs i.e., the intersection of the three DSLs. Then, the reusable language modules regard the pseudostates and the triggers.

\begin{figure}
\centering
\includegraphics[width=1\linewidth]{images/puzzle-modularization.pdf}
\caption{Results for the state machines case study: extracting language modules}
\label{fig:puzzle-modularization}
\end{figure}

\subsection{Evaluating relevance: Identifying potential reuse in the wild}

The second part of the evaluation of our approach is intended to identify potential reuse in the wild. To do so, we explore the public GitHub repositories and download all the DSLs that fit with the technological space and language workbench we used in our approach. The result of this exploration is a set of metamodels that we can use. However, the semantic part of the DSLs is not that easy since kermeta 3 is a novel approach that is under construction. In any case, we consider that analyzing potential reuse at the level of the syntax is a good insight to know if there is potential reuse.

Once we have the set of metamodels (about 2.800), we build a matrix. 

The experiments were conducted using a version of \toolname implemented in Java. Further, 
\toolname was installed in the Grid5000 Cloud, which is a cluster with more than 5000 cores from were we took XX dual-CPU Dell Blades with Intel Xeon X3470 CPUs running at 2.93GHz, with 16 threads 
per CPU, and CentOS v6. Each dual-CPU Dell Blade has 36GB of RAM. 

%\begin{table*}[htbp]
%  \centering
% \scalebox{0.8}{
%\begin{tabular}{|p{0.2\textwidth}|p{0.3\textwidth}p{0.1\textwidth}p{0.4\textwidth}|}
%\hline
%\multicolumn{4}{|c|}{\textbf{Hypotheses of Experiment 1}} \\ \hline
%\textbf{Null Hypothesis ($H_0$)} & \multicolumn{ 3}{|p{0.8\textwidth}|}{\toolname is capable of detecting %commonalities in the case study that motivated this research.} \\ \hline
%\textbf{Alt. Hypothesis ($H_1$)} & \multicolumn{ 3}{|p{0.8\textwidth}|}{
%\toolname is not capable of detecting commonalities in the case study that motivated this research.} \\ %\hline
%\textbf{Dependent variable} & \multicolumn{ 3}{|p{0.8\textwidth}|}{The set of ecores representing our %languages. }\\ \hline
%\textbf{Blocking variables} & \multicolumn{ 3}{|p{0.8\textwidth}|}{The most sold phones and the market %share indexes. }\\ \hline
%\textbf{Model used as input} & \multicolumn{ 3}{|p{0.8\textwidth}|}{\textit{models in %\url{urlhacialosmodelos}} }
%\\
%\hline \hline


%\multicolumn{4}{|c|}{\textbf{Hypotheses of Experiment 2}} \\ \hline
%\textbf{Null Hypothesis ($H_0$)} & \multicolumn{ 3}{|p{0.8\textwidth}|}{The use of \toolname 
%will not result in a higher market-share impact metric than selecting the most commonly sold 
%phones, for a given maximum budget.} \\ \hline
%\textbf{Alt. Hypothesis ($H_1$)} & \multicolumn{ 3}{|p{0.8\textwidth}|}{The use of \toolname 
%will result in a higher market-share impact metric than selecting the most commonly sold 
%phones, for a given maximum budget.}\\ \hline
%\textbf{Model used as input} & \multicolumn{ 3}{|p{0.8\textwidth}|}{\textit{Android feature model presented %in Figure \ref{fig:featureModel}} }\\ 
%\hline
%% @J - Do you mean independent?! My understanding is that a blocking variable is a grouping variable...
%\textbf{Blocking variables} & \multicolumn{ 3}{|p{0.8\textwidth}|}{The most sold phones, market share %indexes and the maximum cost allowed set to 600\$. }\\ \hline
%\textbf{Model used as input} & \multicolumn{ 3}{|p{0.8\textwidth}|}{\textit{Android feature model presented in Figure \ref{fig:featureModel}} }\\
%\hline \hline

%\multicolumn{4}{|c|}{\textbf{Constants}} \\ \hline
%\textbf{CSP solver} & \multicolumn{ 3}{|p{0.8\textwidth}|}{\textit{ChocoSolver v2} } \\ \hline
%\textbf{Heuristic for variable selection in the CSP solver} & \multicolumn{ 3}{|p{0.8\textwidth}|}{\textit{Default}}\\
%\hline 

%\hline 
%\end{tabular}%
%}
%\caption{Hypotheses and design of experiments.}
%  \label{tab:Exp1aDesign}
%\end{table*}
%\todo{poner la tabla para con los datos de los experimentos que vamos a ejecutar/hemos ejecutado. Intenta pensar cuales pueden ser las conclusiones que quieres extraer. Yo propongo 3 abajo.}

%Table \ref{tab:Exp1aDesign} shows the hypothesis of the experiments executed to validate our 
%approach. To make the experiments reproducible, a number of fixed assumptions are made, such as homogeneous feature costs. ChocoSolver 
%\footnote{\url{http://www.emn.fr/z-info/choco-solver/}}, with it's default heuristic, is 
%used as the CSP solver for extracting software products from the feature model presented 
%in Figure \ref{fig:featureModel}

%\textbf{Technological space and experimental platform:} Currently, there are diverse techniques available for the implementation of syntax and semantics of DSLs \cite{Mernik:2005b}. Language designers can, for example, choose between using context-free grammars or metamodels as specification formalism for syntax. Similarly, there are at least three methods for expressing semantics: operationally, denotationally, and axiomatically \cite{Mosses:2001}. In this paper we are interested on DSLs which syntax is specified by means of metamodels and semantics is specified operationally as a set methods (a.k.a, \textit{domain-specific actions} \cite{Combemale:2013}). Each language construct is specified by means a metaclass and the relationship between language constructs are specified as references between metaclasses. In turn, domain-specific actions are specified as java-like methods that are allocated in each metaclass.

%\section{Tool demonstration}
\label{sec:tooldemo}

The approach presented in this paper is tool-supported. That means that we implemented a the tooling needed to support the ideas and concepts. The tooling is implemented on top of Eclipse IDE and, in particular, the Eclipse Modeling Framework. In the following, we present two different videos. The first one illustrates the use of our tool in the illustrating example presented in section \ref{sec:motivation}. The second video shows the tool for the case study of the state machines presented in section \ref{sec:validation}.

\begin{itemize}
\item \textbf{Tool demonstration 1:} FSM, Logo, and Flowcharts. URL: video

\item \textbf{Tool demonstration 2:} UML state diagrams, Rhapsody, and Harel state charts. URL: video
\end{itemize}

%\section{Threads to validity}
\label{sec:threadstovalidity}

The most important thread to validity of our tool corresponds to the constraint that all the members of the family must be implemented in the same technology. We are aware that this constraints limits the applicability of our tool since it is possible to find families of DSLs where each member is implemented in a different tool. However, we have two points to discuss to this regard.

First, the problem we are trying to solve has to be with adaptation of the same DSL. As we mentioned in the introduction, one of the situations where families of DSLs appear is where an initial language is adapted for fit a particular domain. Under this situation, we can expect that the technologies are the same. 

Second, the contribution of this paper is not only limited to the tool itself. All the ideas expressed here with respect to the issues that are important to analyze in a family of DSLs are important as well. We consider that our tool can be conceived for other technologies. 
%\section{Related work and discussion}
\label{sec:relatedwork}

Leveraging reuse in the construction of DSLs is an objective that has been previously discussed in the research community on software languages engineering. One of the very first achievements towards this objective was the notion of components-based language development. With the time passing, approaches are becoming more sophisticate, thus supporting more complex modularization scenarios, and being applicable for more diverse technological spaces \cite{Mernik:2013,Rumpe:2010,Voelter:2013b}. 

More recent approaches are focused not only on dealing with modularization issues, but also on facilitating the reuse process itself. For instance, Melange \cite{Degueule:2015} is a tool-supported approach that introduces some operators (such as slice, inheritance, and merge) intended to manipulate legacy DSLs in such a way that they can be easily integrated in new developments. Using Melange, a language designer can combine a set of DSLs in different ways to produce a new one.

The main contribution of our approach is that it advances towards the automation of the reuse process. We demonstrate that, with the correct abstractions and assuming some constraints, the process can be automated by means of reverse-engineering techniques. It is important mentioning that our approach uses some of the ideas presented by Caldiera and Basili \cite{Caldiera:1991}. That approach proposes reverse-engineering methodologies for extracting reusable modules in objects oriented software. In addition, our work is based on an observation about commonalities and potential reuse provided by V\"oelter et al \cite[p. 60-61]{voelter:2013}. We show that the notion of commonalities is quite useful for extracting reusable language modules.

There are, however, some open issues that need further investigation. In particular, during the conduction of this research, and trying to apply the approach in further case studies, we realized that the comparison operators to detect those commonalities can become an Achilles' heel. In some cases, the notion of commonality can be associated to a given \textit{functionality} (abstractly speaking), more than equality in the specification. For example, there are many DSLs that use constraints languages but that use different language constructs to support them. They share the functionality of constraint languages, but the specifications do not match. That does not mean that there is not potential reuse. The detection of this kind of commonalities can become quite difficult due to the ambiguity in that notion of functionality.

As a matter of fact, considering more flexible approaches for the detection of commonalities can have additional advantages. Consider for example two DSL that define different constraints languages where one is better defined (more completely or with more powerful capabilities) than the other. If this situations can be detected, and language designers can chose a preferred language module, our approach can become useful not only to achieve reuse but also to improve quality of existing DSLs. We claim that more complex overlapping identification (probably with human intervention) should be provided.

%Software reuse has been largely studied during the last decades. The complexity behind the definition of reusable software is well-known and many approaches has been proposed to facilitate this task. The work of Caldiera and Basili \cite{Caldiera:1991} is a clear example of such approaches. It is intended to automatically extract reusable software modules from existing software systems. Then, a set of metrics for qualifying the quality of the produced modules. As mentioned in the introduction, this work has inspired the approach presented in that work paper. 

%The main contribution of the research presented this paper is that we apply those ideas in the construction of DSLs. To do so, we use a strategy based on Venn diagrams that, in turn, has been inspired in the phenomenon of domains overlapping identified by .

%As a matter of fact, our approach is not the first one that tries to increase reuse in the construction of DSLs. The community of software languages engineering has been intensively working on this issue and noways we can find approaches for components-based languages development (such as ). In that context, our approach can be positioned as a reverse engineering technique for increasing reuse where reusable language modules are extracted from existing DSLs. It is worth noting that there are other approaches working on reverse engineering for DSLs. For example, the research presented in \cite{vacchi:2014} and \cite{Kuhn:2015}, is intended to synthesize language product lines (i.e., software product lines where the products are DSLs) from a DSL specification. 

 


%\section{Conclusions}
\label{sec:conclusions}

In this paper, we provide an approach for exploiting reuse during the construction of DSLs. We demonstrate that it is possible to partially automate the reuse process by identifying commonalities among DSLs and automatically extracting reusable language modules that can be later used in the construction of new DSLs. We evaluated our approach in a real industrial case study and we demonstrate that there is an important amount of potential reuse in DSLs in public repositories.

%As future work, we plan to propose approaches to automatically build language product lines i.e., software product lines where the products are DSLs. The intention is to follow with the idea of automating the reuse process. This time, using ideas that facilitate the management of the variability existing among DSLs. 

\section*{Acknowledgments}
The research presented in this paper is supported by the European Union within the FP7 Marie Curie Initial Training Network ``RELATE" under grant agreement number 264840 and VaryMDE, a collaboration between Thales and INRIA.

% BIBLIOGRAPHY
\bibliographystyle{abbrv}
\bibliography{0-main}

\end{document}
